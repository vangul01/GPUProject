%%%%%%%%%%%%%%%%%%%%%%%%%%%%%%%%%%%%%%%%%%%%%%%%%%%%%%%%%%%%%%%%%%%%%%%%%%%%%%%%
%2345678901234567890123456789012345678901234567890123456789012345678901234567890
%        1         2         3         4         5         6         7         8

\documentclass[letterpaper, 10 pt, conference]{ieeeconf}  % Comment this line out
                                                          % if you need a4paper
%\documentclass[a4paper, 10pt, conference]{ieeeconf}      % Use this line for a4
                                                          % paper

\IEEEoverridecommandlockouts                              % This command is only
                                                          % needed if you want to
                                                          % use the \thanks command
\overrideIEEEmargins
% See the \addtolength command later in the file to balance the column lengths
% on the last page of the document



% The following packages can be found on http:\\www.ctan.org
%\usepackage{graphics} % for pdf, bitmapped graphics files
%\usepackage{epsfig} % for postscript graphics files
%\usepackage{mathptmx} % assumes new font selection scheme installed
%\usepackage{times} % assumes new font selection scheme installed
%\usepackage{amsmath} % assumes amsmath package installed
%\usepackage{amssymb}  % assumes amsmath package installed

\title{\LARGE \bf
Parallelizing Logistic Regression
}

%\author{ \parbox{3 in}{\centering Huibert Kwakernaak*
%         \thanks{*Use the $\backslash$thanks command to put information here}\\
%         Faculty of Electrical Engineering, Mathematics and Computer Science\\
%         University of Twente\\
%         7500 AE Enschede, The Netherlands\\
%         {\tt\small h.kwakernaak@autsubmit.com}}
%         \hspace*{ 0.5 in}
%         \parbox{3 in}{ \centering Pradeep Misra**
%         \thanks{**The footnote marks may be inserted manually}\\
%        Department of Electrical Engineering \\
%         Wright State University\\
%         Dayton, OH 45435, USA\\
%         {\tt\small pmisra@cs.wright.edu}}
%}

\author{Wansang Lim$$ and Valerie Angulo$$% <-this % stops a space
%\thanks{*This work was not supported by any organization}% <-this % stops a space
%\thanks{$^{1}$H. Kwakernaak is with Faculty of Electrical Engineering, Mathematics and Computer Science,
%        University of Twente, 7500 AE Enschede, The Netherlands
%        {\tt\small h.kwakernaak at papercept.net}}%
%\thanks{$^{2}$P. Misra is with the Department of Electrical Engineering, Wright State University,
%        Dayton, OH 45435, USA
%        {\tt\small p.misra at ieee.org}}%
}


\begin{document}



\maketitle
\thispagestyle{empty}
\pagestyle{empty}


%%%%%%%%%%%%%%%%%%%%%%%%%%%%%%%%%%%%%%%%%%%%%%%%%%%%%%%%%%%%%%%%%%%%%%%%%%%%%%%%
\begin{abstract}
  Logistic regression is a statistical analysis tool used as a predictive analytic in a variety of disciplines. In this paper, we focus on parallelizing binary logistic regression analysis for predictive analytics in data science for environmental science datasets. Parallelizing logistic regression would improve computation time and decrease memory latency when analyzing large data sets, allowing for more data to be processed faster. In this paper, we compare a sequential implementation of binary logistic regression with a CUDA implementation, a parallelized R implementation and a multiprocessor OpenCV version, looking at the relationship between dataset sizes and time taken to process the data. Our findings point to improvements in computation time and less memory latency for CUDA versions of logistic regression.

\end{abstract}


%%%%%%%%%%%%%%%%%%%%%%%%%%%%%%%%%%%%%%%%%%%%%%%%%%%%%%%%%%%%%%%%%%%%%%%%%%%%%%%%
\section{INTRODUCTION}

Logistic Regression is used as a predictive analytic in many disciplines ranging from biology and conservation to business. It models a binary dependent and one or more binary or nonbinary independent variables. This is useful in cases of observing phenomena that may occur due to a specific event. The purpose of logistic regression is to predict the occurrence of phenomena based on acquired current data. Mathematically, the binary dependent variable is either 0 or 1 and indicates the presence or absence of a certain condition, such as alive/dead or win/lose, that may be related to the independent conditions. In this paper, we are primarily concerned with the applications of binary logistic regression analysis for predictive analytics in data science, particularly for environmental science datasets. For our purpose, logistic regression is used to classify dependent variables into different groups. 

Currently, there is an abundance of large datasets that are open sourced and easily accessible to the public. This is especially beneficial for scientific research. However, processing large data sets is time consuming and resource intensive for CPU in terms of memory and computation time. Sequential implementations of logistic regression require a lot of time to process smaller amounts of data and have a high memory latency. Implementation of a parallelized binary logistic regression would allow for an increased amount of data to be processed in less time and with less resource intensive computations. Logistic regression is an excellent analytic to parallelize because it primarily utilizes matrix multiplication, which is easy to convert to parallel code. It also utilizes an inverse function which is a variation of matrix multiplication and determines the natural log of a matrix, a function that is easily supported by parallelism. In this paper, we compare a sequential implementation of binary logistic regression with a CUDA implementation, a parallelized R implementation and a multiprocessor OpenCV version, looking at the relationship between dataset sizes and time taken to process the data. Our findings point to improvements in computation time and less memory latency for CUDA versions of logistic regression, as well as parallelized R implementations.


\section{BACKGROUND}

Logistic regression is a predictive analytic that is used to categorize data into different groups. It can be binomial, ordinal or multinomial and is based on linear regression in that logistic regression estimates a multiple linear regression function. It is the correct regression to use if there is a binary dependent variable, for example time spent studying versus whether a student passes or fails a course, or other situations depending on independent variables such as win/lose, dead/alive, yes/no, present/absent etc. Binary logistic regression is used to describe the relationship between a binary dependent variable and one or many independent variables. This predictive analytic tool is useful in data science for the discovery of new insights as to what patterns one might expect with certain independent factors at play. 

Logistic regression model is useful in determining the relationship between a random value and its covariants, so our data was randomly generated values and we looked to find a dependent Y.

We first coded a sequential version of a linear regression and then modified it to be a logistic regression. In this linear model example
$$l = B0 + B1x1 + B2x2$$
the x’s are the predictors and the Bi’s are the parameters of the model, with B0 being a constant term. To turn this equation into a logistic model, we must take the log odds of the linear regression like so.
$$o = b^ B0 + B1x1 + Bx2$$
b is the base of the logarithm and exponent. 


Logistic regression’s core is estimating the log odds of an event. The log odds of a value is a linear combination of one or more independent variables, which can be either binary or any real value. With binary logistic regression, we can predict the odds of a dependent variable based on the independent variables/predictors/x values. The odds are determined by dividing the probability of an outcome occurring by the probability that is will not occur. We then determine a continuous version of the dependent binary by taking the logarithm of the odds of an event happening. The log of the odds is the logit of the probability.

To fit the logit of the probability of success with the predictors/x values, we must take the inverse of the natural log, then we are able to obtain a continuous predictor for the odds of an event happening. 

\section{LITERATURE SURVEY}

We researched a variety of papers regarding parallelizing logistic regression, as well as the applications of parallelized and sequential logistic regression.

\subsection{What have the others do to solve this problem}
Add stuff

\subsection{What are the pros and cons of this previous work}
Add stuff

\subsection{Implementations from others that we’ve used or referenced} 
Add stuff

\subsection{Units}

\begin{itemize}

\item blank
\item blank
\item blank
\item Use a zero before decimal points: Ò0.25Ó, not Ò.25Ó. Use Òcm3Ó, not ÒccÓ. (bullet list)

\end{itemize}


\subsection{Equations}


$$
\alpha + \beta = \chi \eqno{(1)}
$$

Note that the equation is centered using a center tab stop. Be sure that the symbols in your equation have been defined before or immediately following the equation. Use Ò(1)Ó, not ÒEq. (1)Ó or Òequation (1)Ó, except at the beginning of a sentence: ÒEquation (1) is . . .Ó

\begin{itemize}


\item blank
\item blank
\item blank
\item blank

\end{itemize}


\section{PROPOSED SOLUTION}
We propose to parallelize the logistic regression in order to improve predictive behavior in a number of fields. This will allow researchers to 

\section{EXPERIMENTAL SETUP}
For our experiment, we are using a binary logistic regression, where we are dealing with a dependent variable that can only be either 0 or 1, whereas our independent variables are real numbers. To make sure our generated data set was good, we followed the assumptions of data used for logistic regression applications such as that the dependent variable is binary, there are no outliers in the data or strong correlations between independent data sets and that there were no values present in the data that were below -3.29 or above 3.29(https://www.statisticssolutions.com/what-is-logistic-regression/). 

Another consideration we had to take into account was the model fit, having more independent variables increases the amount of variance ($$R^2$$) but adding too many variables will decrease the generalizability of the predictive analytic, rendering it less accurate. To analyze the accuracy of our logistic regression, we computed its goodness-of-fit based on the Chi square test, as well as the amount of variance $$R^2$$.  

Logistic regression model is useful in determining the relationship between a random value and its covariants, so our data was randomly generated values and we looked to find a dependent Y.

We first coded a sequential version of a linear regression and then modified it to be a logistic regression. 


\subsection{blank}

blank

\subsection{Figures and Tables}

Positioning Figures and Tables: Place figures and tables at the top and bottom of columns. Avoid placing them in the middle of columns. Large figures and tables may span across both columns. Figure captions should be below the figures; table heads should appear above the tables. Insert figures and tables after they are cited in the text. Use the abbreviation ÒFig. 1Ó, even at the beginning of a sentence.

\begin{table}[h]
\caption{An Example of a Table}
\label{table_example}
\begin{center}
\begin{tabular}{|c||c|}
\hline
One & Two\\
\hline
Three & Four\\
\hline
\end{tabular}
\end{center}
\end{table}


   \begin{figure}[thpb]
      \centering
      \framebox{\parbox{3in}{We suggest that you use a text box to insert a graphic (which is ideally a 300 dpi TIFF or EPS file, with all fonts embedded) because, in an document, this method is somewhat more stable than directly inserting a picture.
}}
      %\includegraphics[scale=1.0]{figurefile}
      \caption{Inductance of oscillation winding on amorphous
       magnetic core versus DC bias magnetic field}
      \label{figurelabel}
   \end{figure}
   

Figure Labels: Use 8 point Times New Roman for Figure labels. Use words rather than symbols or abbreviations when writing Figure axis labels to avoid confusing the reader. As an example, write the quantity ÒMagnetizationÓ, or ÒMagnetization, MÓ, not just ÒMÓ. If including units in the label, present them within parentheses. Do not label axes only with units. In the example, write ÒMagnetization (A/m)Ó or ÒMagnetization {A[m(1)]}Ó, not just ÒA/mÓ. Do not label axes with a ratio of quantities and units. For example, write ÒTemperature (K)Ó, not ÒTemperature/K.Ó


\section{RESULTS AND DISCUSSION}
add stuff

\section{CONCLUSIONS}

A conclusion section is not required. Although a conclusion may review the main points of the paper, do not replicate the abstract as the conclusion. A conclusion might elaborate on the importance of the work or suggest 
applications and extensions. 

\begin{itemize}

\item final bullets 
\item final bullets 
\item final bullets 
\item final bullets 
\end{itemize}

\addtolength{\textheight}{-12cm}   % This command serves to balance the column lengths
                                  % on the last page of the document manually. It shortens
                                  % the textheight of the last page by a suitable amount.
                                  % This command does not take effect until the next page
                                  % so it should come on the page before the last. Make
                                  % sure that you do not shorten the textheight too much.

%%%%%%%%%%%%%%%%%%%%%%%%%%%%%%%%%%%%%%%%%%%%%%%%%%%%%%%%%%%%%%%%%%%%%%%%%%%%%%%%



%%%%%%%%%%%%%%%%%%%%%%%%%%%%%%%%%%%%%%%%%%%%%%%%%%%%%%%%%%%%%%%%%%%%%%%%%%%%%%%%



%%%%%%%%%%%%%%%%%%%%%%%%%%%%%%%%%%%%%%%%%%%%%%%%%%%%%%%%%%%%%%%%%%%%%%%%%%%%%%%%
%\section*{APPENDIX}

%Appendixes should appear before the acknowledgment.

\section*{ACKNOWLEDGMENT}

Thanks to Professor Zahran for all your help.

\begin{thebibliography}{99}

\bibitem{c1} G. O. Young, ÒSynthetic structure of industrial plastics (Book style with paper title and editor),Ó 	in Plastics, 2nd ed. vol. 3, J. Peters, Ed.  New York: McGraw-Hill, 1964, pp. 15Ð64.
\bibitem{c2} W.-K. Chen, Linear Networks and Systems (Book style).	Belmont, CA: Wadsworth, 1993, pp. 123Ð135.
\bibitem{c3} H. Poor, An Introduction to Signal Detection and Estimation.   New York: Springer-Verlag, 1985, ch. 4.
\bibitem{c4} B. Smith, ÒAn approach to graphs of linear forms (Unpublished work style),Ó unpublished.
\bibitem{c5} E. H. Miller, ÒA note on reflector arrays (Periodical styleÑAccepted for publication),Ó IEEE Trans. Antennas Propagat., to be publised.
\bibitem{c6} J. Wang, ÒFundamentals of erbium-doped fiber amplifiers arrays (Periodical styleÑSubmitted for publication),Ó IEEE J. Quantum Electron., submitted for publication.
\bibitem{c7} C. J. Kaufman, Rocky Mountain Research Lab., Boulder, CO, private communication, May 1995.
\bibitem{c8} Y. Yorozu, M. Hirano, K. Oka, and Y. Tagawa, ÒElectron spectroscopy studies on magneto-optical media and plastic substrate interfaces(Translation Journals style),Ó IEEE Transl. J. Magn.Jpn., vol. 2, Aug. 1987, pp. 740Ð741 [Dig. 9th Annu. Conf. Magnetics Japan, 1982, p. 301].
\bibitem{c9} M. Young, The Techincal Writers Handbook.  Mill Valley, CA: University Science, 1989.
\bibitem{c10} J. U. Duncombe, ÒInfrared navigationÑPart I: An assessment of feasibility (Periodical style),Ó IEEE Trans. Electron Devices, vol. ED-11, pp. 34Ð39, Jan. 1959.
\bibitem{c11} S. Chen, B. Mulgrew, and P. M. Grant, ÒA clustering technique for digital communications channel equalization using radial basis function networks,Ó IEEE Trans. Neural Networks, vol. 4, pp. 570Ð578, July 1993.
\bibitem{c12} R. W. Lucky, ÒAutomatic equalization for digital communication,Ó Bell Syst. Tech. J., vol. 44, no. 4, pp. 547Ð588, Apr. 1965.
\bibitem{c13} S. P. Bingulac, ÒOn the compatibility of adaptive controllers (Published Conference Proceedings style),Ó in Proc. 4th Annu. Allerton Conf. Circuits and Systems Theory, New York, 1994, pp. 8Ð16.
\bibitem{c14} G. R. Faulhaber, ÒDesign of service systems with priority reservation,Ó in Conf. Rec. 1995 IEEE Int. Conf. Communications, pp. 3Ð8.
\bibitem{c15} W. D. Doyle, ÒMagnetization reversal in films with biaxial anisotropy,Ó in 1987 Proc. INTERMAG Conf., pp. 2.2-1Ð2.2-6.
\bibitem{c16} G. W. Juette and L. E. Zeffanella, ÒRadio noise currents n short sections on bundle conductors (Presented Conference Paper style),Ó presented at the IEEE Summer power Meeting, Dallas, TX, June 22Ð27, 1990, Paper 90 SM 690-0 PWRS.
\bibitem{c17} J. G. Kreifeldt, ÒAn analysis of surface-detected EMG as an amplitude-modulated noise,Ó presented at the 1989 Int. Conf. Medicine and Biological Engineering, Chicago, IL.
\bibitem{c18} J. Williams, ÒNarrow-band analyzer (Thesis or Dissertation style),Ó Ph.D. dissertation, Dept. Elect. Eng., Harvard Univ., Cambridge, MA, 1993. 
\bibitem{c19} N. Kawasaki, ÒParametric study of thermal and chemical nonequilibrium nozzle flow,Ó M.S. thesis, Dept. Electron. Eng., Osaka Univ., Osaka, Japan, 1993.
\bibitem{c20} J. P. Wilkinson, ÒNonlinear resonant circuit devices (Patent style),Ó U.S. Patent 3 624 12, July 16, 1990. 






\end{thebibliography}




\end{document}
